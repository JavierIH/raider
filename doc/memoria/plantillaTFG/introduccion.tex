\chapter{Introducci�n}

En este cap�tulo...

%section es un apartado dentro de un chapter. Tambi�n existe subsection y subsubsection
\section{Estado del arte}

%las referencias a art�culos se ponen con \cite, 
%las referencias a im�genes \ref, 
%y las referencias a ecuaciones \eqref

Este tema.... Esto es un ejemplo de cita de un art�culo \cite{gonzalezeducational}.

%itemize es una lista. Cada t�rmino lleva delante un \item
\begin{itemize}
\item \textbf{ejemplo de lista de puntos}. Ejemplo.
\item \textbf{ejemplo2 de lista}. Ejemplo2.
\end{itemize} 

Ejemplo de referencia a figura (figura \ref{uc3m}).

%caption es el pie de foto, y label es el nombre que se da a la imagen para referenciarla despu�s. label no puede llevar acentos y no se muestra de cara al documento final (es s�lo interno).
\begin{figure}[h]
\centering
\includegraphics[width=0.45\textwidth]{figuras/uc3m}   
\caption{Logotipo de la UC3M \copyright UC3M}
\label{uc3m}
\end{figure}


\section{Motivaci�n del proyecto}

La idea...

\section{Estructura del documento}

A continuaci�n y para facilitar la lectura del documento, se detalla el contenido de cada cap�tulo.

\begin{itemize}
\item En el cap�tulo 1 se realiza una introducci�n.
\item En el cap�tulo 2 se hace un repaso...
\end{itemize}
