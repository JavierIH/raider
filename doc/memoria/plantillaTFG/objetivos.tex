%chapter introduce un nuevo cap�tulo

\chapter{Objetivos}

gmmgfngfngfngfngfngfnhgf
\begin{itemize}
\item \textbf{ejemplo de lista de puntos}.

Cocoloco loco loco loco tacacacaca
\item \textbf{ejemplo de lista de puntos}.\\Cocoloco loco loco loco tacacacaca
\item \textbf{ejemplo2 de lista}.
\end{itemize}

\section{Implantaci�n de un controlador basado en visi�n por computador}

El objetivo principal de este proyecto consiste en la implantaci�n de un controlador para un robot humanoide que proporcione las capacidades necesarias para el desarrollo de algoritmos de interacci�n con el entorno basados en t�nicas de visi�n por computador.

De este modo, se pretende dotar al robot de un comportamiento aut�nomo en ejercicios de interacci�n con el entorno tales como la navegaci�n y la manipulaci�n de objetos f�sicos ( TODO : Pegarle un guantazo a otro robot).

Mas especificamente, el desarrollo se divide en dos tareas:

\begin{itemize}
\item \textbf{Sistema de visi�n}.

Cocoloco loco loco loco tacacacaca
\item \textbf{Implementaci�n de OpenCV}.
\end{itemize}
 

\section{Estudio de sensores y actuadores}

ok

\section{Dise�o electr�nico}

ok

\section{Dise�o mec�nico}

ok

\section{Programaci�n}

ok



\paragraph{Palabras clave:} palabraclave1, palabraclave2, palabraclave3.
