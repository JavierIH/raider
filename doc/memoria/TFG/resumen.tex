%chapter introduce un nuevo cap�tulo
\chapter{Resumen}

En este proyecto se ha desarrollado la plataforma rob�tica mini-humanoide RAIDER, (Robot Antropom�rfico para la Investigaci�n y Desarrollo en Entornos Reales) con capacidad para actuar de forma aut�noma bas�ndose en algoritmos de visi�n por computador. Para ello se ha escogido un robot mini-humanoide comercial del que se han extra�do sus servomotores, y se ha dise�ado una nueva estructura con piezas imprimibles en una impresora 3D. Sobre el robot, se ha integrado un sistema de procesamiento de im�genes formado por una c�mara USB y un controlador desarrollado sobre un ordenador de tama�o reducido, la BeagleBone Black. Tambi�n se ha realizado un estudio de sensores y actuadores aptos para su montaje en un robot mini-humanoide. Entre ellos se han seleccionado aquellos que aumentan las capacidades del robot.

\medskip Tras fabricar y montar las nuevas piezas se ha procedido a programar el robot. Se han seguido tres lineas: Locomoci�n, sensorizaci�n y visi�n. En la programaci�n de la locomoci�n se presentan los pasos que se han seguido desde el movimiento de una articulaci�n simple hasta la combinacion de estos movimientos para producir movimientos mas complejos como la caminata o el control del equilibrio.En el apartado de visi�n, se han estudiado y desarrollado t�cnicas de path planning basadas en la b�squeda de trayectorias mediante la detecci�n y esquelitizaci�n del espacio navegable basada en el algoritmo de Zhang-Suen. Adicionalmente, se han programado otras funciones como el tracking de una pelota o la lectura de c�digos qr. Por �ltimo, en la sensorizaci�n, se han hecho diferentes librer�as para el control desde la BeagleBone Black de los sensores que se montan en el robot y la extracci�n de datos �tiles que apoyan a la parte de visi�n.
El procesamiento de im�genes se ha combinado con la informaci�n recibida por los sensores para dise�ar aplicaciones aptas para la competici�n en CEABOT y otros eventos.


\paragraph{Palabras clave:} palabraclave1, palabraclave2, palabraclave3.

\chapter{Abstract}

(El resumen en ingles)

\paragraph{Keywords:} keyword1, keyword2, keyword3.