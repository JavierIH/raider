%chapter introduce un nuevo cap�tulo
\chapter{Resumen}

En este proyecto se ha desarrollado la plataforma rob�tica mini-humanoide RAIDER, (Robot Antropom�rfico para la Investigaci�n y Desarrollo en Entornos Reales) con capacidad para actuar de forma aut�noma bas�ndose en algoritmos de visi�n por computador. Para ello, se ha escogida una base comercial sobre la que se ha realizado un redise�o mec�nico completo con la ayuda de una impresora 3D. Sobre el robot, se ha montado un SBC (Single Board Computer), que ha permitido desarrollar algoritmos de visi�n por computador con las librer�as de OpenCV y ZBar. Tamb�en se han a�adido diferentes sensores y actuadores adicionales para aumentar la versatilidad del robot.

\medskip Con el robot, se han programado las funciones de locomoci�n necesarias para dotar a la plataforma rob�tica de movilidad absoluta, incluyendo marcha b�peda y control de ca�das.

\medskip Adicionalmente, se han realizado funciones basadas en visi�n como la b�squeda de trayectorias de navegaci�n en entornos complejos, la detecci�n de lineas y el an�lisis de c�digos QR. Estas funciones han servido como base para el dise�o aplicaciones de competici�n orientadas a presentar a RAIDER al concurso nacional de robots mini-humanoides CEABOT.


\paragraph{Palabras clave:} rob�tica, visi�n artificial, mini-humanoide.

\chapter{Abstract}

(El resumen en ingles)

\paragraph{Keywords:} keyword1, keyword2, keyword3.