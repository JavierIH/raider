%chapter introduce un nuevo cap�tulo
\chapter{Presupuesto}

El presupuesto del proyecto puede consultarse en el anexo 6. Para abarcar las partes esenciales del proyecto, se ha dividido en cuatro secciones.

\medskip La primera secci�n es el presupuesto de personal. En esta categor�a se han estudiado tanto las horas invertidas en el proyecto como su renumeraci�n.

\medskip La segunda secci�n incluye el gasto en equipos. Se han utilizado dos componentes esenciales: Un ordenador port�til personal y una impresora 3D. Ambos se han utilizado para el proyecto, aunque no se adquirieron exclusivamente para ello.

\medskip En tercer lugar se encuentra la secci�n de materiales. Esta secci�n incluye componentes diversos; desde sensores y controladoras hasta pl�stico y tornillos. Todo el conjunto supone la lista de materiales necesarios para construir un robot completo. Tanto en este apartado como en el anterior se ha realizado un c�lculo de la amortizaci�n en base a la siguiente f�rmula:

\[ \frac{A}{B} \cdot C \cdot D \]

Donde:


\medskip $A$ son los meses durante los que el equipo es utilizado.

$B$ es el periodo de depreciaci�n en meses.

$C$ es el coste del equipo (sin incluir IVA).

$D$ es el porcentaje de uso que se dedica al proyecto.

\bigskip Por �ltimo, el apartado de subcontrataci�n de tareas incluye los gastos derivados del trabajo de terceros. En este caso se ha inclu�do el rutado de la placa de expansi�n mediante maquinaria CNC.
