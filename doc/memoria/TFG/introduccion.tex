\chapter{Introducci�n}

En este cap�tulo...

%section es un apartado dentro de un chapter. Tambi�n existe subsection y subsubsection
\section{XXXXXXXXXX}

%las referencias a art�culos se ponen con \cite, 
%las referencias a im�genes \ref, 
%y las referencias a ecuaciones \eqref

Este tema.... Esto es un ejemplo de cita de un art�culo \cite{gonzalezeducational}.

%itemize es una lista. Cada t�rmino lleva delante un \item
\begin{itemize}
\item \textbf{ejemplo de lista de puntos}. Ejemplo.
\item \textbf{ejemplo2 de lista}. Ejemplo2.
\end{itemize} 

Ejemplo de referencia a figura (figura \ref{uc3m}).

%caption es el pie de foto, y label es el nombre que se da a la imagen para referenciarla despu�s. label no puede llevar acentos y no se muestra de cara al documento final (es s�lo interno).
\begin{figure}[h]
\centering
\includegraphics[width=0.45\textwidth]{figuras/uc3m}   
\caption{Logotipo de la UC3M \copyright UC3M}
\label{uc3m}
\end{figure}


\section{XXXXXXXXXXXX}

La idea...
\section{Asociaci�n de Rob�tica de la Universidad Carlos III}

La Asociaci�n de Rob�tica de la Universidad Carlos III de Madrid, AsRob, surgi� en el a�o 2006 ( <- TODO a�o real ) con el objetivo de acercar la rob�tica a los alumnos de la universidad que compart�an inquietudes e inter�s por el campo de la rob�tica.

A d�a de hoy, la asociaci�n cuenta con mas de cien ( <- TODO cuantos ) miembros activos repartidos en cinco lineas de investigaci�n independientes, como son:
\begin{itemize}
\item \textbf{Veh�culos A�reos no Tripulados (UAVs)}.
\item \textbf{Robot Devastation}.
\item \textbf{Robots Personales de Competici�n}.
\item \textbf{Robots Mini-Humanoides}.
\item \textbf{Impresoras 3D Open-Source}.
\end{itemize}

Sin embargo, cabe destacar que aunque se trata de proyectos diferentes, existe una gran sinergia entre ellos. Particularmente, los miembros de la linea de Robots Mini-Humanoides, estamos muy ligados al estudio de las impresoras 3D, investigando diferentes t�cnicas de impresi�n, dise�o de estructuras y materiales. Ejemplo de ello es el proyecto MYOD ( <- TODO referencia ), en el que se propone la construcci�n de robots mini-humanoides compuestos integramente con piezas impresas y replicables.


\section{Descripci�n del proyecto}

blablablabla mi proyecto es la monda

\section{Estructura del documento}

A continuaci�n y para facilitar la lectura del documento, se detalla el contenido de cada cap�tulo.

\begin{itemize}
\item En el cap�tulo 1 se realiza una introducci�n.
\item En el cap�tulo 2 se hace un repaso...
\end{itemize}
