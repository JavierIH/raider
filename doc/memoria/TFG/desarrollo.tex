%chapter introduce un nuevo cap�tulo
\chapter{Desarrollo}

Este proyecto de resume en.................

\section{Dise�o de las partes mec�nicas}
\subsection{Cabeza}
\subsection{Tronco}
\subsection{Brazos}
\subsection{Piernas}



\section{Montaje del controlador}
\subsection{Esquema de montaje} %Muy Alto nivel
\subsection{Sistema de alimentaci�n}
\subsection{Adecuaci�n de sensores}
\subsection{Desarrollo de una placa de expansi�n}
\subsection{Puesta en marcha del controlador}
\section{Programaci�n}
\subsection{Sistema de locomoci�n}

Antes de comenzar con esta secci�n, es importante se�alar algunos detalles importantes. Si bien es cierto, el control del robot girar� en torno a un algoritmo de visi�n, su funcionamiento de apoyar� en un control de locomoci�n robusto que permita al robot realizar desplazamientos seguros sobre el terreno. Por tanto, dado que se trata de un robot b�pedo, la programaci�n de movimientos no es un tema trivial como lo podr�a ser en un robot con ruedas.

Para conseguir que Raider pueda moverse con soltura se han seguido varios pasos, cada cual de un nivel superior al anterior.

\subsubsection{Movimiento de los actuadores Dyamixel}

Para comunicarse con los actuadores Dynamixel (de los que hablamos en TODO ) debemos utilizar su protocolo particular. Los servos son controlados mediante el envio de paquetes de datos binarios. Existen dos tipos de paquetes en el protocolo: Los paquetes de instrucciones, que son los que envia el controlador a los servos; y los paquetes de estado, que son los los servos env�an al controlador.

Cada servo tiene una ID, o dicho de otra forma, un n�mero de identidad propio e irrepetible que identifica a un servo particular dentro del bus. La comunicaci�n en el bus se realiza mediante el intercambio de paquetes de instrucciones y estados con una ID concreta.
Por esta raz�n, en un mismo bus no deben existir servos con la misma ID, ya que provocar�n colisiones entre los paquetes e impedir�n el correcto funcionamiento del sistema. Sin embargo, estas ID son facilmente reprogramables y pueden modificarse realizando una escritura sobre el registro 3 (0X03).

El protocolo de comunicaci�n utilizado es una comunicaci�n serie as�ncrona de 8 bits, con 1 bit de Stop y sin paridad.

Half duplex UART is a serial communication protocol where both TxD and RxD cannot be used at the same time. This method is generally used when many devices need to be connected to a single bus. Since more than one device are connected to the same bus, all the other devices need to be in input mode while one device is transmitting. The Main Controller that controllers the Dynamixel actuators sets the communication direction to input mode, and only when it is transmitting an Instruction Packet, it changes the direction to output mode.





poseen una tabla de registros (tabla) sobre la cual podemos variar varios par�metros referente a su estado y su funcionamiento. La tabla de registros puede consultarse en TODO .

Para realizar una rotaci�n en un servomotor, ser�a suficiente con escribir en el registro 32, Goal Position,





\subsubsection{Movimiento sincronizado de las articulaciones}
\subsubsection{Funciones de movimientos combinados} 
\subsubsection{Creaci�n de movimientos completos} 
\subsubsection{Controlador de mocimiento} 
\subsection{Algoritmos de visi�n} 
\subsection{Pruebas de CEABOT} 





















\paragraph{Palabras clave:} palabraclave1, palabraclave2, palabraclave3.

