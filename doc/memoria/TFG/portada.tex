\begin{titlepage}
\begin{center}

%forma de introducir im�genes. el \\[0.5 cm] de final de l�nea introduce un salto de ese tama�o.
%width=1\textwidth indica el tama�o de la im�gen (valores entre 0-1). 
 \includegraphics[width=1\textwidth]{figuras/cabecera.png}  \\[0.5 cm]

\large \textsc{Departamento de Ingenier�a de Sistemas y Autom�tica} \\ [1 cm]

\large TRABAJO FIN DE GRADO\\[1 cm]

\huge \textsc{Desarrollo de una plataforma rob�tica mini-humanoide con visi�n artificial TODO }\\[7 cm]

%flushleft alinea a la izquierda el texto
\begin{flushleft} \Large
\emph{Autor:} Javier Isabel Hern�ndez\\[0.5 cm]
\emph{Director:} Felix Rodr�gez Ca�adillas \\
\emph{Tutor:} Alberto Jard�n Huete
\end{flushleft}

%rellena de blanco el resto de la p�gina para escribir abajo del todo
\vfill

% Bottom of the page
{\large Legan�s, Septiembre 2014}

\end{center}
\end{titlepage}