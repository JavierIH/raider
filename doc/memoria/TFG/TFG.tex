% Plantilla realizada por Santiago Morante Cendrero
% Tuneada por Javi

%parametros de tipo libro
\documentclass[10pt,a4paper]{book}

%idioma espa�ol y acentos
\usepackage[spanish]{babel}
\usepackage[latin1]{inputenc}

%algunos s�mbolos matem�ticos y paquete para usar subim�genes
\usepackage{amsmath}
\usepackage{amsfonts}
\usepackage{amssymb}
\usepackage{graphicx}
\usepackage{subfigure}
\usepackage{listings}
\usepackage{appendix}

%para generar �ndice con hiperv�nculos
\usepackage{hyperref}

%parametros del documento (sus propiedades)
\hypersetup{
    pdftitle={Nombre del alumno - TFG - a�o},
    pdfsubject={TFG - a�o},
    pdfauthor={Nombre del alumno},
    pdfkeywords={palabraclave1} {palabraclave2} {palabraclave3},
    colorlinks,
    citecolor=black,
    filecolor=black,
    linkcolor=black,
    urlcolor=black,
}

%empieza el documento
\begin{document}  

%elementos antes del trabajo en s� se meten dentro de frontmatter
\frontmatter

%cada incluye referencia a un archivo de tipo .tex
\begin{titlepage}
\begin{center}

%forma de introducir im�genes. el \\[0.5 cm] de final de l�nea introduce un salto de ese tama�o.
%width=1\textwidth indica el tama�o de la im�gen (valores entre 0-1). 
 \includegraphics[width=1\textwidth]{figuras/cabecera.png}  \\[0.5 cm]

\large \textsc{Departamento de Ingenier�a de Sistemas y Autom�tica} \\ [1 cm]

\large TRABAJO FIN DE GRADO\\[1 cm]

\huge \textsc{Desarrollo de una plataforma rob�tica mini-humanoide con visi�n artificial TODO }\\[7 cm]

%flushleft alinea a la izquierda el texto
\begin{flushleft} \Large
\emph{Autor:} Javier Isabel Hern�ndez\\[0.5 cm]
\emph{Director:} Felix Rodr�gez Ca�adillas \\
\emph{Tutor:} Alberto Jard�n Huete
\end{flushleft}

%rellena de blanco el resto de la p�gina para escribir abajo del todo
\vfill

% Bottom of the page
{\large Legan�s, Septiembre 2014}

\end{center}
\end{titlepage}

\begin{flushleft}

Copyright \copyright  2014. Javier Isabel Hern�ndez

Esta obra est� sujeta a la licencia Reconocimiento-NoComercial-CompartirIgual 4.0 Internacional de Creative Commons. Para ver una copia de esta licencia, visite http://creativecommons.org/licenses/by-nc-sa/4.0/.


\end{flushleft}

\cleardoublepage

\begin{flushleft} \large
\textbf{T�tulo:} Desarrollo de una plataforma rob�tica mini-humanoide con visi�n artificial \\
\textbf{Autor:} Javier Isabel Hern�ndez\\
\textbf{Director:} F�lix Rodr�guez Ca�adillas \\ 
\textbf{Tutor:} Alberto Jard�n Huete\\ [1 cm]

\end{flushleft} 

\begin{center} \LARGE
EL TRIBUNAL \\ [1 cm]
\end{center}

\begin{flushleft} \LARGE
Presidente: \\ [1 cm]
Vocal: \\ [1 cm]
Secretario: \\ [1.5 cm]
\end{flushleft}

\large
Realizado el acto de defensa y lectura del Trabajo Fin de Grado el d�a ....... de ....................   de ... en .........., en la Escuela Polit�cnica Superior de la Universidad Carlos III de Madrid, acuerda otorgarle la CALIFICACI�N de: \\ [2 cm]

\begin{center}
 \large VOCAL \\ [2.2 cm]
\end{center}

\begin{minipage}{0.5\textwidth}
 \begin{flushleft}
 \large SECRETARIO
\end{flushleft}
\end{minipage}
\begin{minipage}{0.5\textwidth}
\begin{flushright}
 \large PRESIDENTE
\end{flushright} 
\end{minipage}

\chapter{Agradecimientos}

En primer lugar me gustar�a agradecer a mi tutor, Alberto Jard�n, y a mi director F�lix Rodr�guez, el haberme dado la oportunidad de realizar un trabajo de final de grado que me apasiona.

Quiero dar las gracias a Miguel Gonz�lez-Fierro, por darme tan buenos consejos TODO



Quiero dar las las gracias de todo coraz�n a mi familia, a mi padre Javier, a mi madre Miriam y a mi hermana Marta. A ellos que siempre me han apoyado en todas las decisiones que he dedicido tomar, me han animado a perseguir mis objetivos y me han consolado en los momentos mas dif�ciles.

Por �timo, quiero dar las gracias a mi novia Silvia, que ha estado a mi lado anim�ndome siempre que lo he necesitado, y a quien he prometido que llevar� a la playa en cuanto termine este proyecto. 




%chapter introduce un nuevo cap�tulo
\chapter{Resumen}




En este proyecto se ha desarrollado la plataforma rob�tica mini-humanoide Raider, (Robot Antropom�rfico para la Investigaci�n y Desarrollo en Entornos Reales) con capacidad para actuar de forma aut�noma bas�ndose en algoritmos de visi�n por computador. Para ello se ha dise�ado una configuraci

 integrado en el robot mini-humanoide un sistema de procesamiento de im�genes formado por una c�mara USB y un controlador desarrollado sobre un ordenador de tama�o reducido, As� como diferentes sensores que apoyen a la parte de vision.( TODO mejorar el t�rmino )
Tras dise�ar, fabricar y montar las nuevas piezas se ha procedido a programar el robot. En la programaci�n de la locomoci�n se presentan los pasos que se han seguido desde el movimiento de una articulaci�n simple hasta la combinacion de estos movimientos para producir movimientos mas complejos como la caminata o el control del equilibrio. Por otra parte, se han estudiado y desarrollado algoritmos de visi�n en los que el robot basar� su comportamiento. Mas espec�ficamente, se han desarrollado t�cnicas de path planning basadas en la b�squeda de trayectorias mediante la detecci�n y esquelitizaci�n del espacio navegable basada en el algoritmo de de ZANG SHUEN. Adicionalmente, se han programado otras funciones como el tracking de una pelota o la lectura de c�digos qr.

El procesamiento de im�genes se ha combinado con la informaci�n recibida por los sensores para dise�ar aplicaciones aptas para la competici�n en CEABOT y otros eventos. Para concluir el proyecto, el robot se ha presentado a la edici�n de 2014 de CEABOT.

El proyecto queda como una plataforma viable sobre la que realizar nuevos proyectos por sus capacidades y robustez.



\paragraph{Palabras clave:} palabraclave1, palabraclave2, palabraclave3.

\chapter{Abstract}

(El resumen en ingles)

\paragraph{Keywords:} keyword1, keyword2, keyword3.

%genera �ndice
\tableofcontents

%�ndice de figuras. Tambi�n se podr�a hacer uno de tablas (listoftables)
\listoffigures
 
%empieza la parte descriptiva del trabajo
\mainmatter
 
\chapter{Introducci�n}

En este cap�tulo...

%section es un apartado dentro de un chapter. Tambi�n existe subsection y subsubsection
\section{Estado del arte}

%las referencias a art�culos se ponen con \cite, 
%las referencias a im�genes \ref, 
%y las referencias a ecuaciones \eqref

Este tema.... Esto es un ejemplo de cita de un art�culo \cite{gonzalezeducational}.

%itemize es una lista. Cada t�rmino lleva delante un \item
\begin{itemize}
\item \textbf{ejemplo de lista de puntos}. Ejemplo.
\item \textbf{ejemplo2 de lista}. Ejemplo2.
\end{itemize} 

Ejemplo de referencia a figura (figura \ref{uc3m}).

%caption es el pie de foto, y label es el nombre que se da a la imagen para referenciarla despu�s. label no puede llevar acentos y no se muestra de cara al documento final (es s�lo interno).
\begin{figure}[h]
\centering
\includegraphics[width=0.45\textwidth]{figuras/uc3m}   
\caption{Logotipo de la UC3M \copyright UC3M}
\label{uc3m}
\end{figure}


\section{Motivaci�n del proyecto}

La idea...

\section{Estructura del documento}

A continuaci�n y para facilitar la lectura del documento, se detalla el contenido de cada cap�tulo.

\begin{itemize}
\item En el cap�tulo 1 se realiza una introducci�n.
\item En el cap�tulo 2 se hace un repaso...
\end{itemize}


%chapter introduce un nuevo cap�tulo
\chapter{Descripci�n de la linea de investigaci�n}

TODO

\paragraph{Palabras clave:} palabraclave1, palabraclave2, palabraclave3.
 %Descripcion de la linea de investigacion

%chapter introduce un nuevo cap�tulo
\chapter{Estado del arte}

Este proyecto de resume en.................

\paragraph{Palabras clave:} palabraclave1, palabraclave2, palabraclave3.
 %Estado del arte

%chapter introduce un nuevo cap�tulo

\chapter{Objetivos}

gmmgfngfngfngfngfngfnhgf
\begin{itemize}
\item \textbf{ejemplo de lista de puntos}.

Cocoloco loco loco loco tacacacaca
\item \textbf{ejemplo de lista de puntos}.\\Cocoloco loco loco loco tacacacaca
\item \textbf{ejemplo2 de lista}.
\end{itemize}

\section{Implantaci�n de un controlador basado en visi�n por computador}

El objetivo principal de este proyecto consiste en la implantaci�n de un controlador para un robot humanoide que proporcione las capacidades necesarias para el desarrollo de algoritmos de interacci�n con el entorno basados en t�nicas de visi�n por computador.

De este modo, se pretende dotar al robot de un comportamiento aut�nomo en ejercicios de interacci�n con el entorno tales como la navegaci�n y la manipulaci�n de objetos f�sicos ( TODO : Pegarle un guantazo a otro robot).

Mas especificamente, el desarrollo se divide en dos tareas:

\begin{itemize}
\item \textbf{Sistema de visi�n}.

Cocoloco loco loco loco tacacacaca
\item \textbf{Implementaci�n de OpenCV}.
\end{itemize}
 

\section{Estudio de sensores y actuadores}

ok

\section{Dise�o electr�nico}

ok

\section{Dise�o mec�nico}

ok

\section{Programaci�n}

ok



\paragraph{Palabras clave:} palabraclave1, palabraclave2, palabraclave3.


%chapter introduce un nuevo cap�tulo
\chapter{Elecci�n de componentes}\label{chaptercomponentes}

Como punto de partida, se han seleccionado una serie de componentes adecuados para dotar al robot con las capacidades necesarias para cumplir los objetivos. Dado que el sistema se va a redise�ar completamente, se elegir�n los componentes que mejor se adapten a nuestras necesidades sin tener que ce�irse a limitaciones de compatibilidad.

\section{Plataforma rob�tica}

El primer paso para la realizaci�n de este proyecto fue el estudio y selecci�n de las plataformas rob�ticas que se encuentran en el mercado actualmente. Dado que el objetivo es encontrar un robot humanoide sobre el que se pueda implantar un sistema de visi�n, es necesario analizar diversos aspectos; algunos mec�nicos como el numero y fuerza de los actuadores, y otros electr�nicos como la capacidad de procesamiento y velocidad del controlador. Sin embargo, dado que este proyecto es autofinanciado en su mayor medida, el factor econ�mico tambi�n es un limitante destacable. Buscamos una plataforma que cumpla los siguientes requisitos:

\begin{itemize}
\item{Programable en C/C++.} 

Se requiere que sea programable en C/C++ y que adem�s no dependa de una IDE concreta.

\item{Posibilidad de conectarle una c�mara.}

Se necesita que permita conectarle una c�mara y realizar programas con OpenCV.

\item{Expandible con sensores y actuadores.}

El sistema debe permitir la adici�n de nuevos sensores y actuadores, sin verse limitado por hardware o software.

\medskip \item{Servos de al menos 12kg/cm.}

Ya que se van a incluir nuevas partes, es absolutamente necesario que los servos puedan soportar carga adicional colocada en el robot.

\item{Chasis reconfigurable.}

Aunque no es tan importante como el resto, es posible que el robot requiera modificaciones, por lo que una base reconfigurable y vers�til ser� apreciada.

\end{itemize}

En el siguiente apartado se presenta un estudio las principales opciones.



\bigskip \subsubsection{Comparativa y elecci�n final}

Realizando un primer an�lisis, ninguno de los robots candidatos cumple los requ�sitos. Todos ellos obligan a utilizar entornos de desarrollo y lenguajes propios para su programaci�n. El Robonova y el Robovie tienen unos servos demasiado d�biles, lo que dificultar�a mucho a�adir m�s peso al robot. Entre el Kondo y el Bioloid, se ha elegido el Bioloid por tres razones: Es m�s f�cil de modificar, el kit contiene m�s servos  y es m�s barato. Tambi�n, previendo las modificaciones futuras, se contempl� la idea de comprar �nicamente los servos Dynamixel que usa el Bioloid por separado. Sin embargo, result� m�s econ�mico adquirir el kit completo de Bioloid Comprehensive.


\section{Modificaciones estructurales}

El Bioloid Comprehensive es un buen punto de partida, sin embargo tiene algunos puntos d�biles que conviene revisar. Adem�s, para poder implantar en el robot los dispositivos que requiere este proyecto se necesitar�n mejorar las capacidades de la plataforma.


\subsection{Cabeza m�vil}

Un requisito importante del proyecto es permitir que la c�mara que vamos a montar pueda moverse con libertad para enfocar a diferentes zonas de su entorno. Dado que en CEABOT la mayor�a de los datos  que aporta el entorno est�n situados en el suelo, necesitamos que la c�mara al menos pueda dirigirse hacia el frente y hacia el suelo. Esto lo conseguiremos con la adici�n de un microservo PWM y una plataforma articulada para la cabeza. Dadas las caracter�sticas de este movimiento, no necesitamos un servo con grandes capacidades, ya que su rango de acci�n estar� muy limitado y su efecto ser� despreciable en el reparto de pesos del robot.

\begin{figure}[h]
\centering
\includegraphics[width=0.5\textwidth]{figuras/microservo}   
\caption{Microservo PWM.}
\label{microservo}
\end{figure}
\FloatBarrier


Se ha elegido un microservo PWM por su bajo tama�o y peso, su bajo precio y su sencillez a la hora de montarlo y programarlo. El principio de funcionamiento de un servo PWM es muy simple. De las tres patillas de su conector, dos son de alimentaci�n y la tercera se encarga de recibir una se�al PWM que, variando la amplitud de su pulso, ordena al servo a moverse a una posici�n fijada.

Particularmente, se ha escogido un servo Tower Pro MG90s como el de la figura \ref{microservo}, cuyas especificaciones presentan un torque de 2.4=kg/cm, y una transmisi�n met�lica soportada por rodamientos. Este �ltimo dato es muy importande si tenemos en cuenta que el sistema de cabeza m�vil se situar� en una zona extrema del robot, y que un golpe provocado por una caida forzar� de forma directa el eje del servo de la articulaci�n. Este servo proporcionar� a la cabeza la robustez necesaria para salir indemne en este tipo de accidentes. Accidentes que por otra parte son muy comunes teniendo en cuenta que el robot estar� destinado a la realizaci�n de pruebas de competici�n.

\subsection{Cintura m�vil}\label{subsubcintura}

Otra de las modificaciones b�sicas a realizar sobre la plataforma rob�tica, ha sido la inclusi�n de un servo adicional Dynamixel AX-12A para articular la cintura. A parte de la mejora de capacidades que se produce en los movimientos del robot, servir� para girar la direcci�n de la c�mara radialmente. Podr�a decirse que entre el servo de la cabeza y el de la cintura, se ha creado un sistema distribuido de pan-tilt que permitir� mover la c�mara en todas las direcciones manteniendo fija la base del robot, es decir, sus piernas.

\section{Sensorizaci�n}

El funcionamiento del robot y su control va a estar basado principalmente en la c�mara y los algoritmos de visi�n que se programar�n. No obstante, la c�mara consume muchos recursos del procesador y existen alguna tareas sencillas que es m�s f�cil programar con sensores m�s simples. A continuaci�n se muestra un estudio de sensores que pueden ser �tiles en el proyecto.

\subsection{Sensores de distancia}

Existen diferentes tipo de sensores de distancia. El prop�sito de estos sensores no es otro que la medici�n de longitudes utilizando diferentes principios f�sicos. Se ha planteado el uso de sensores infrarrojos y/o de sensores de ultrasonidos.

\subsubsection{Sensores de luz infrarroja}

Los sensores infrarrojos, como el que aparece en al figura \ref{sensorir} basan su funcionamiento en la reflexi�n de luz infrarroja sobre superficies. El sensor est� formado por dos LEDs infrarrojos, uno emisor y otro receptor. El emisor emite de forma continua una luz infrarroja dirigida hacia un punto fijo. Si en ese punto fijo se encuentra un objeto f�sico, la luz se reflejar� y ser� captada por el diodo receptor. La salida de estos sensores se mide mediante la diferencia de potencial que se produce en el diodo receptor. Se ha estudiado la inclusi�n en el proyecto de sensores infrarrojos Sharp, especialmente de su modelo GP2Y0A21YK, que tiene un rango de operaci�n de entre 4 y 150cm.

\begin{figure}[h]
\centering
\includegraphics[width=0.5\textwidth]{figuras/sharp}   
\caption{Sensor infrarrojo.}
\label{sensorir}
\end{figure}
\FloatBarrier

\subsubsection{Sensores de ultrasonidos}

Los sensores de ultrasonidos detectan distancias basandose en el tiempo en el que un sonido de alta frecuencia recorre el espacio. Normalmente, los sensores de ultrasonidos tienen dos focos, uno emisor y otro receptor. Una de las diferencias de este tipo de sensores con los sensores infrarrojos es que mientras los sensores infrarrojos toman medidas de un punto, el rango de operaci�n de los sensores de ultrasonidos se abre en un como de 60� de amplitud desde su emisor. Se ha analizado el sensor de ultrasonidos HC-SR04 como posible candidato para formar parte del robot por su f�cil accesibilidad. Otro punto importante es que en estos sensores la salida es proporcional a la distancia medida, es decir, se adec�a a una recta. Es por esto por lo que podremos realizar medidas reales directamente midiendo la salida del sensor.

\begin{figure}[h]
\centering
\includegraphics[width=0.5\textwidth]{figuras/sensorus}   
\caption{Sensor de ultrasonidos}
\label{sensorus}
\end{figure}
\FloatBarrier

\subsubsection{Experimentos de medida}

Para poner en pr�ctica y obeservar las diferencias te�ricas que existen entre el funcionamiento de ambos sensores, se ha realizado un peque�o experimento en el que se han realizado medidas de longitud entre el sensor y un obst�culo. Para ello, se han analizado tres configuraciones posibles. En cada im�gen, se ha marcado en azul el rango de acci�n del sensor. La linea roja representa la distancia que medir� el sensor en cada situaci�n.

\medskip En el experimento de la figura \ref{experimento1}, se ha colocado un obst�culo como los que se utilizan en la prueba de navegaci�n de CEABOT en posici�n frontal y recta. Puede observarse que en esta situaci�n ambos sensores medir�an de forma similar la distancia entre el sensor y el obst�culo.

\begin{figure}[h]
\centering
\begin{tabular}{ >{\centering\arraybackslash}m{0.5\textwidth} >{\arraybackslash}m{0.5\textwidth}}
\includegraphics[width=0.45\textwidth]{figuras/expir1} & \includegraphics[width=0.45\textwidth]{figuras/expus1}  \\
\end{tabular}
\caption{Primer experimento.}
\label{experimento1}
\end{figure} 

En el experimento de la figura \ref{experimento2} se ha desplazado el objeto ligeramente hacia la derecha. Puede observarse c�mo el rango de acci�n del sensor infrarrojo, al tener una forma lineal, no detecta el obst�culo. Mientras tanto, el sensor de ultrasonidos es capaz de detectarlo, ya que en esa posici�n sigue estando dentro de su rango de acci�n. 

\begin{figure}[h]
\centering
\begin{tabular}{ >{\centering\arraybackslash}m{0.5\textwidth} >{\arraybackslash}m{0.5\textwidth}}
\includegraphics[width=0.45\textwidth]{figuras/expir2} & \includegraphics[width=0.45\textwidth]{figuras/expus2}  \\
\end{tabular}
\caption{Segundo experimento.}
\label{experimento2}
\end{figure} 
\FloatBarrier

Por �ltimo, el experimento de la figura \ref{experimento3} muestra el comportamiento de ambos sensores al enfrentarse a una superficie oblicua. Tal y como puede observarse en la imagen, el sensor infrarrojo realiza una medida puntual a la zona del obst�culo que se interpone en su rango de acci�n. Por otra parte, el sensor de ultrasonidos nos devuelve la distancia del punto del objeto que, encontr�ndose dentro de su rango, constituye la distancia m�nia entre ambos.

\begin{figure}[h]
\centering
\begin{tabular}{ >{\centering\arraybackslash}m{0.5\textwidth} >{\arraybackslash}m{0.5\textwidth}}
\includegraphics[width=0.45\textwidth]{figuras/expir3} & \includegraphics[width=0.45\textwidth]{figuras/expus3}  \\
\end{tabular}
\caption{Tercer experimento.}
\label{experimento3}
\end{figure} 

De este experimento pueden sacarse varias conclusiones. La primera de ellas es que no existe un sensor mejor que el otro, cada uno destaca en una funcionalidad. Un sensor infrarrojo nos dar� una mejor precisi�n a la hora de realizar medidas de distancia en un punto concreto, sin embargo, su rango de acci�n es limitado y ser� importante controlar donde est� apuntando exactamente en cada momento. El sensor de ultrasonidos por su parte, detectar� un obst�culo con mayor probabilidad, pero siempre existir� un incertidumbre respecto al posicionamiento exacto del obst�culo que estamos midiendo. De esta forma, un sensor infrarrojo ser� �til cuando necesitemos precisi�n en la posici�n del objeto y por este motivo, ser� el seleccionado para montarse en el robot. De todos modos, no se descarta la idea de montar tambi�n sensores de ultrasonidos en un futuro.

\subsection{Sensores inerciales}

El control del equilibrio del robot es muy importante en configuraciones b�pedas, ya que estos robots son conocidos por su facilidad para tropezar y caer al suelo. Con el objetivo de analizar la posici�n del robot respecto al suelo se ha requerido la inclusi�n de diferentes sensores inerciales.

\subsubsection{Aceler�metro y giroscopio}

En el mercado existen varios modelos econ�micos. En este proyecto se ha utilizado el sensor MPU9150 que aparece enla figura \ref{imu}. Se trata de un sensor inercial compuesto de aceler�metro de 3 ejes, giroscopio de 3 ejes y br�jula de 3 ejes. Este sensor combina dos sensores, un MPU6050 (incluye el acelerometro y el giroscopio) y un AK8975 (incluye la br�jula). Sin embargo, en la pr�ctica se observ� que la br�jula incluida en el sensor era muy propensa a sufrir interferencias, por lo que se decidi� montar una br�jula adicional que no tuviese estos problemas.

\begin{figure}[h]
\centering
\includegraphics[width=0.5\textwidth]{figuras/imu}   
\caption{Sensor inercial MPU9150.}
\label{imu}
\end{figure}

\subsubsection{Br�jula magn�tica}

La inclusi�n de una br�jula es indispensable cuando de requiere conocer la direcci�n en la que se desplaza un robot. En este proyecto se montar� un br�jula para evaluar si el desplazamiento del robot se produce de forma recta y efectuar los redireccionamientos necesarios. Existen diferentes modelos de br�julas digitales en el mercado. Todas ellas, basan su funcionamiento en la detecci�n y medici�n de campos magn�ticos. Para lograr conocer la orientaci�n del robot, la br�jula deber� apoyarse en su posici�n respecto a la del campo magn�tico terrestre. El gran problema de estos dispositivos es su facilidad para modificar sus medidas cuando existe un campo magn�tico fuerte en su entorno. Esta interferencias pueden ser causadas por aparatos electr�nicos, estructuras met�licas o incluso por los propios componentes del robot.

\begin{figure}[h]
\centering
\includegraphics[width=0.5\textwidth]{figuras/compass}   
\caption{Br�jula magn�tica CMPS03.}
\label{compass}
\end{figure}
\FloatBarrier

\medskip Para este proyecto se ha utilizado una br�jula CMPS03 como la de la figura \ref{compass}. El motivo de haber seleccionado este modelo frente a otros m�s com�nes, como por ejemplo la HMC5883L, se debe a que ya ha sido utilizada en anteriores proyectos rob�ticos en la asociaci�n con muy buenos resultados. Esto se debe en parte a que la placa del sensor incluye un PIC que se encarga de calibrar y comunicar los datos medidos, de forma que no ser� necesario que el controlador del robot realice estos c�lculos y podr� utilizar directamente la medida extra�da.

\subsection{C�mara}

La elecci�n de la c�mara ha sido condicionada principalmente por su compatibilidad con el driver de Linux v4l2. Existe una amplia variedad de c�maras v�lidas en el mercado. Dado el car�cter de este proyecto, se ha optado por utilizar una webcam, ya que son m�s accesibles que las c�maras avanzadas de investigaci�n y los algoritmos de visi�n que ser�n programados no requieren im�genes de alta calidad.

\begin{figure}[h]
\centering
\includegraphics[width=0.5\textwidth]{figuras/camara}   
\caption{Microsoft LifeCam Cinema.}
\label{camara}
\end{figure}
\FloatBarrier


\medskip El modelo selecionado ha sido una Microsoft LifeCam Cinema como la que aparece en la figura \ref{camara}. Entre sus car�cter�sticas destacan su posibilidad de grabar video en 720p HD, enfoque autom�tico y la regulaci�n de brillos en tiempo real. Adicionalmente, se trata de una c�mara de f�cil desmontaje y dimensiones contenidas, por lo que ser� sencillo integrarla en el robot.

\medskip Cabe remarcar que antes de utilizar esta c�mara se realizaron pruebas con una Hama Pocket, pero dada su baja calidad de construcci�n present� diversos problemas de fiabilidad y fue despreciada en favor de la Microsoft. 

\section{Elecci�n del controlador}

Llegamos a un punto cr�tico del proyecto, y es la elecci�n de un controlador para nuestro sistema. La controlador CM-5 contenida en el kit original de Bioloid Comprehensive se encuentra muy lejos de poder soportar el conjunto de nuevos dispositivos que se usar�n en el proyecto. Necesitaremos reemplazarla por un sistema m�s complejo que nos permita conectar y programar los sensores seleccionados, programar algoritmos de visi�n y comunicarse con los servos Dynamixel.

\medskip Para solucionar esto se decide colocar un SBC (Single Board Computer, ordenador de placa �nica), que dadas sus capacidades de procesamiento se utilizar� como controlador principal. As� mismo, ser� el encargado de realizar el procesamiento de im�genes. Junto al SBC, se conectar� un controlador m�s sencillo basado en un microcontrolador para controlar los Dynamixel AX-12A por separado. 

\medskip De esta forma, el controlador del robot estar� formado por dos elementos, el SBC que se encargar� del procesamiento de im�genenes y sensores, y un microcontrolador se encargar� de la locomoci�n del robot.

\subsection{Controlador de locomoci�n}

El controlador de locomoci�n debe encargarse de mover los 20 servos del robot y de comunicarse con el controlador de visi�n para recibir instrucciones. A continuaci�n se realiza un estudio de posibles placas que podr�an utilizarse.

\subsubsection{Arduino}

Las placas Arduino se han hecho famosas por su f�cil programaci�n y puesta en marcha. Dada su popularidad, existe una gran comunidad de usuarios que generan documentaci�n de forma constante. Con una placa Arduino disponemos de una amplia colecci�n de bibliotecas para interactuar con sensores y actuadores. Sin embargo, estas placas por si solas no soportan la comunicaci�n con los servos Dynamixel. Por esta raz�n, se ha desestimado su uso en el proyecto.  

\subsubsection{Arbotix}

La placa Arbotix, mostrada en la figura \ref{arbotix} est� basada, al igual que las placas Arduino, en un microcontrolador AVR. No obstante, esta placa de desarrollo est� orientada principalmente a la construcci�n de peque�os robots, por lo que posee capacidades mejores que las de las placas Arduino. Entre sus especificaciones destacan: 2 puertos serie (uno de ellos reservado para el bus Dynamixel), 32 pines de entrada/salida, 8 entradas anal�gicas, 2 drivers de 1A para motores de continua y z�calos para la conexi�n de m�dulos XBEE. 

\begin{figure}[h]
\centering
\includegraphics[width=0.5\textwidth]{figuras/arbotix}   
\caption{Placa Arbotix.}
\label{arbotix}
\end{figure}
\FloatBarrier


El problema de estas placas es que no existe un distribuidor que las venda en Europa. Eso no solo significa que sean dif�ciles de conseguir, sino tambi�n que no tienen demasiados usuarios, y por tanto la documentaci�n es escasa y heterogenea.

\subsubsection{Serie OpenCM}

Recientemente la empresa Robotis ha sacado a la venta una nueva linea de placas de desarrollo, las placas OpenCM. En la figura \ref{cm900} se muestra una CM900. Estas placas de desarrollo est�n basadas en las placas Arduino, copiando sus funcionalidades y a�adiendo al sistema la posibilidad de comunicarse con actuadores Dynamixel. Su entorno de programaci�n est� basado en Arduino IDE y continene bibliotecas similares a las de Arduino para la comunicaci�n con sensores y actuadores. Su alta compatibilidad con Arduino proporciona al usuario una mayor facilidad a la hora de buscar documentaci�n sobre c�mo utilizar la placa.

\begin{figure}[h]
\centering
\includegraphics[width=0.5\textwidth]{figuras/cm900}   
\caption{Placa CM900.}
\label{cm900}
\end{figure}
\FloatBarrier

Si bien es cierto, es necesario comentar que la primera edici�n, la CM900, present� problemas en su dise�o que la conviertieron en una placa muy poco fiable y propensa a deteriorarse prematuramente. En este proyecto se utilizar� la OpenCM9.04, sucesora de la CM900, con similares capacidades y tama�o reducido. Entre sus especificaciones destacan: Un microcontrolador ARM Cortex-M3, 26 pines de entrada/salida, 10 entradas analogicas de 12 bits y 3 puertos serie (uno de ellos reservado para el bus Dynamixel).

\begin{figure}[h]
\centering
\includegraphics[width=0.6\textwidth]{figuras/opencm}   
\caption{Placa OpenCM9.04.}
\label{opencm}
\end{figure}
\FloatBarrier

\subsection{Controlador de visi�n}

Como controlador principal, se utilizar� un SBC (Single Board Computer). Por tama�o y capacidades existen dos alternativas directas: La Raspberry Pi B y la BeagleBone Black.

\subsubsection{Raspberry Pi}

La Raspberry Pi es un ordenador completo del tama�o de una tarjeta de cr�dito, dirigida a formar parte de proyectos de electr�nica, informatica y rob�tica. Dado su bajo coste y f�cil adquisici�n, la Raspberry Pi cuenta con una comunidad de usuarios muy extensa repartida a lo largo del mundo. En internet existen colecciones de tutoriales educativos para formar al usuario y dar soporte a sus proyectos. En el cuadro \ref{raspec} se describen las especificaciones t�cnicas de la placa.

\begin{table}[h]
\centering
\begin{tabular}{ >{\arraybackslash}m{4.2cm} >{\arraybackslash}m{5.8cm}}
\hline
Especificaciones Raspberry PI modelo B \\
\hline \hline
\medskip CPU \medskip & ARM 1176JZF\-S 700 MHz  \\ \hline
\medskip Memoria (SDRAM) \medskip & 512 MB (compartidos con la GPU)\\ \hline
\medskip Puertos USB 2.0 \medskip & 2 (v�a hub USB integrado) \\ \hline
\medskip Almacenamiento integrado \medskip & SD / MMC / ranura para SDIO \\ \hline
\medskip Perif�ricos de bajo nivel \medskip & 8 x GPIO, SPI, I2C, UART \\ \hline
\medskip Consumo energ�tico \medskip & 700 mA (3.5 W) \\ \hline
\medskip Fuente de alimentaci�n \medskip & 5 V v�a Micro USB o GPIO header  \\ \hline
\medskip Dimensiones: \medskip & 85.60mm $\times$ 53.98mm\\ \hline
\medskip Sistemas operativos soportados: \medskip & \medskip GNU/Linux: Debian (Raspbian), Fedora (Pidora), Arch Linux (Arch Linux ARM), Slackware Linux. RISC OS2 \medskip \\ \hline
\end{tabular}
\caption{Especificaciones Raspberry Pi B.}
\label{raspec}
\end{table}
\FloatBarrier


En el caso que nos ocupa, ser�a una perfecta candidata a ocuparse del procesamiento de im�genes del robot, sin embargo hay algunos detalles que es importante conocer. El primer problema de la Raspberry Pi es que la l�gica de sus pines trabaja a 5V, mientras que la placa que hemos seleccionado para la parte de locomoci�n, la OpenCM, funciona a 3.3V. Esto significa que para comunicar ambas placas ser�a necesario incluir un convertidor l�gico en el sistema. Por otra parte, sus GPIOs tienen un funcionamiento puramente binario, sin entradas anal�gicas ni generaci�n de se�ales PWM. Por �ltimo, como puede observarse en la figura \ref{raspberry} aunque tiene un tama�o razonable, la placa cuenta con varios conectores de video, audio, conexion en red... etc que aunque podr�an resultar �tiles en otro tipo de proyectos, en nuestro caso estorbar�n a la hora de integrar el dispositivo en la espalda del robot.

\begin{figure}[h]
\centering
\includegraphics[width=0.5\textwidth]{figuras/raspberry}   
\caption{SBC Raspberry B.}
\label{raspberry}
\end{figure}
\FloatBarrier

\subsubsection{BeagleBone Black}

La BeagleBone Black de Texas Instruments, fotografiada en la figura \ref{beaglebone}, no es tan popular como la Raspberry Pi, sin embargo posee algunas caracter�sticas muy interesantes.

\begin{figure}[h]
\centering
\includegraphics[width=0.5\textwidth]{figuras/bbbboard}   
\caption{SBC BeagleBone Black.}
\label{beaglebone}
\end{figure}
\FloatBarrier


\medskip Como puede observarse en el cuadro \ref{blaspec}, sus caracter�sticas son ligeramente superiores a las de la Raspberry. Principalmente, el hecho de que tenga tal variedad de pines la hace muy adecuada para usarse como controlador principal de un robot. La BeagleBone Black actualmente no cuenta con una comunidad tan extensa como la Raspberry Pi, pero cada vez se ve m�s en proyectos. El problema fundamental de la BeagleBone Black es la falta de bibliotecas en C/C++ que permitan programar sus perif�ricos de bajo nivel. Esto significa que en el caso de utilizar esta placa, se deber�n programar bibliotecas propias para estos prop�sitos. 

\begin{table}[h]
\centering
\begin{tabular}{ >{\arraybackslash}m{4.2cm} >{\arraybackslash}m{5.8cm}}
\hline
Especificaciones BBB \\
\hline \hline
\medskip CPU \medskip  &  AM335x 1GHz ARM Cortex-A8  \\ \hline
\medskip Memoria (SDRAM) \medskip  & 512 MiB \\ \hline
\medskip Puertos USB 2.0 \medskip  & 1x Standard A y 1x mini B \\ \hline
\medskip Almacenamiento integrado \medskip  & 4GB 8-bit eMMC / microSD \\ \hline
\medskip Perif�ricos de bajo nivel \medskip  & \medskip 4xUART, 8x PWM, LCD, GPMC, MMC1, 2x SPI, 2x I2C, A/D Converter, 2x CAN bus \medskip \\ \hline
\medskip Consumo energ�tico \medskip  & 1200 mA (6 W) \\ \hline
\medskip Fuente de alimentaci�n \medskip  & \medskip 5 V v�a Micro USB, jack de 5.5mm a 5 V o GPIO header \medskip \\ \hline
\medskip Dimensiones: \medskip & 86.40 mm $\times$ 53.30 mm \\ \hline
Sistemas operativos soportados: &  \medskip  Fedora, Android, Ubuntu, Debian, openSUSE , �ngstr�m, FreeBSD, NetBSD, OpenBSD, QNX, MINIX 3, RISC OS, y Windows Embedded. \medskip  \\ \hline
\end{tabular}
\caption{Especificaciones BeagleBone Black.}
\label{blaspec}
\end{table}
\FloatBarrier


Por otro lado, su l�gica funciona a 3.3V y es m�s compacta que la Raspberry Pi. Por todo lo comentado, y aunque trabajar con ella pueda resultar m�s tedioso que con la Raspberry Pi, se ha seleccionado la Beaglebone Black como controlador principal del robot.

\subsection{Arquitectura hardware}

Una vez se ha seleccionado todos los elementos que fomar�n parte del robot, se ha dise�ado el conexionado que llevar�n. Tal y como se ha ido viendo a lo largo de este apartado, por un lado tendremos la OpenCM que se encargar� de controlar el servo PWM y los servos Dynamixel. La BeagleBone en cambio, se encargar� de recibir los datos de todos los sensores y de la c�mara. Las dos placas se conectar�n en serie. El diagrama de la figura \ref{dia1} representa las conexiones de todos los dispositivos, a excepci�n de la parte de alimentaci�n, que se ver� en la pr�xima secci�n.

\begin{figure}[h]
\centering
\includegraphics[width=1\textwidth]{figuras/dia1}   
\caption{Diagrama de conexiones.}
\label{dia1}
\end{figure}
\FloatBarrier


\section{Alimentaci�n}

La bater�a contenida en el kit es una bater�a NiMH de 8 elementos AA Sanyo Eneloop que desarrolla 9.6V y 1900mAh. Para el consumo de este robot es una opci�n aceptable que proporciona cerca de 10 minutos de autonom�a. No est� mal si se tiene en cuenta que pruebas con el CEABOT tiene una duraci�n m�xima limitada a 5 minutos. No obstante, tiene un inconveniente, su peso y su volumen es muy alto. Al estar situada en la espalda, el centro de gravedad del robot sube en altura de forma considerable. Esto causa inestabilidades y complica bastante la programaci�n de desplazamientos.

\subsection{Bater�a}

Por las razones descritas, se ha decidido buscar una alternativa. Los requ�sitos para seleccionar una bater�a los marcan tres factores: La tensi�n de funcionamiento de los servos, la de las placas de control y la de los sensores. Pero dado que los servos tienen una tensi�n de operaci�n en un rango de 9V a 12V, respecto al resto de elementos que ser�n alimentados a 5V o 3.3V, nos centraremos principalmente en encontrar una bater�a adecuada para los servos.

\medskip Hoy en d�a, las bater�as NiMH est�n empezando a caer en desuso a favor de las bater�as de litio. Se barajaron dos posibilidades.

\subsubsection{Bater�as de pol�mero de litio}

Las bater�as de pol�mero de litio, m�s conocidas como bater�as LiPo, destacan por desarrollar una tasa de descarga bastante alta y por tener un tama�o y peso reducido. Dado su uso en automodelismo y aeromodelismo, son unas bater�as muy comunes que en los �ltimos a�os han ido bajando su precio. Actualmente existe una gran variedad de bater�as LiPo de diversos tama�os y precio asequible. Cada c�lula LiPo ofrece 3.7V, por lo que la elecci�n correcta para este robot ser�a una bater�a de 3S, o lo que es lo mismo, 11,1V.

\medskip Sin embargo, las bater�as LiPo tienen una desventaja importante, su fragilidad. El proceso de carga y descarga de una bater�a de este tipo no es un asunto trivial, una sobredescarga de la bater�a provoca un deterioro inmediato de la misma, calentando sus elementos e incluso pudiendo llegar a incendiarse. Es por ello que se necesita un cargador con capacidad para balancear las c�lulas, de forma que las tensiones de cada una de ellas se tengan bajo control durante el proceso.

\subsubsection{Bater�as de litio fosfato de hierro}  

Las bater�a de litio fosfato de hierro, es decir, LiFe, son conocidas por su gran resistencia y larga vida �til. Cada c�lula de LiFe desarrolla 3.3V y no presenta problemas por sobredescargas. De hecho, estas bater�as ni siquiera requieren de un cargador balanceador para su carga, sino que pueden cargarse sin problemas con un cargador normal. El gran problema de estas bater�as es que sus elementos son bastante voluminosos, y, al no ser tan populares como las LiPo, tampoco es demasiado f�cil encontrar modelos de tama�o y capacidad adecuados en el mercado. 

\subsubsection{Elecci�n y dise�o del sistema}  

Por su tama�o y prestaciones se ha elegido la bater�a LiPo Rhino de 3S y 1750mAh que aparece en la figura \ref{lipo}. Esta bater�a tiene unas dimensiones perfectas para montarse en la cintura del robot y su peso (109 gramos) es mucho menor al de la bater�a original (215 gramos). 

\medskip A continuaci�n se presenta una tabla (cuadro \ref{consumo}) con una estimaci�n del consumo medio de todos los elementos seleccionados.

\begin{figure}[h]
\centering
\includegraphics[width=0.5\textwidth]{figuras/lipo}   
\caption{Bater�a LiPo.}
\label{lipo}
\end{figure}
\FloatBarrier

\begin{table}[h]
\centering
\begin{tabular}{ >{\arraybackslash}m{2cm} >{\centering\arraybackslash}m{1.3cm}  >{\arraybackslash}m{5cm}  >{\arraybackslash}m{3cm}}
\hline
Dispositivo & N�mero & Consumo unitario & Consumo total  \\
\hline \hline
\medskip Dynamixel AX-12A \medskip & x19  & $11.1V\cdot600mA=6660mW$ & $126540mW$  \\ \hline
\medskip TowerPro MG90s \medskip & x1 & $5V\cdot120mA=600mW$ & $600mW$ \\ \hline
\medskip BeagleBone Black \medskip & x1 & $5V\cdot1200mA=6000mW$ & $6000mW$ \\ \hline
\medskip OpenCM 9.04 \medskip & x1 & $11.1V\cdot90mA\simeq1000mW$ & $1000mW$ \\ \hline 
\medskip Sharp IR \medskip &  x2 & $5V\cdot30mA=150mW$ & $300mW$ \\ \hline
\medskip MPU9150 \medskip & x1 & $3.3V\cdot15mA\simeq50mW$ & $50mW$ \\ \hline
\medskip HMC5883L \medskip &  x1  &  $3.3V\cdot12mA\simeq40mW$ & $40mW$ \\ \hline
\medskip Microsoft LifeCam \medskip & x1  &  $5V\cdot200mA=1000mW$ &  $1000mW$ \\ \hline \hline 
Total  &  &  & \medskip  $135530mW\simeq136W$ \medskip  \\ \hline
\hline
\end{tabular}
\caption{Consumo medio.}
\label{consumo}
\end{table}
\FloatBarrier


Por tanto, la estimaci�n de autonom�a del robot en funcionamiento es la siguiente:
\[ \dfrac{11.1V\cdot1750mAh}{135530mW}\cdot\dfrac{60min}{h}\simeq8.6min\]

Entre 8 y 9 minutos de autonom�a te�ricos es un buena cifra contando con el tama�o del robot y sus prestaciones. Como conclusi�n podemos remarcar el hecho de que gracias al nuevo sistema de alimentaci�n hemos podido introducir los dispositivos necesarios para cumplir los requerimientos de este proyecto sin comprometer la autonom�a del robot.

\subsection{Regulador de tensi�n}

Hasta aqu� parece que es suficiente con la elecci�n de la nueva bater�a, sin embargo, a�n falta un detalle que deberemos resolver. La mayor parte de los dispositivos que forman el conjunto est�n alimentados a 5V. La bater�a que hemos elegido tiene una tensi�n de 11.1V, y se ha observado, que cuando est� cargada al m�ximo desarrolla hasta 12.6V. No podemos alimentar la BeagleBone Black y el resto de componentes a una tensi�n tan alta, por lo que ser� necesario colocar un regulador que nos proporcione 5V.

\subsubsection{Regulador lineal}

La primera idea fue utilizar un peque�o circuito con un regulador lineal LM7805. El LM7805 es un componente muy com�n en los circuitos electr�nicos, ya que ofrecen una salida de 5V a partir de tensiones de entrada de hasta 35V. En principio no parecer�a muy descabellado conectar los servos AX-12A  y la placa OpenCM directamente a la bater�a, y conectar en resto de componentes a trav�s del regulador.

\medskip No obstante existen dos problemas para realizar este montaje. El primero es que la eficiencia de este componente es muy pobre, cercana al 60\% seg�n su hoja de caracter�sticas. Esto provocar�a una disminuci�n dr�stica en el tiempo de operaci�n del robot y una disipaci�n de energia tan alta que ser�a necesario incluir un disipador. Pero el segundo problema es m�s importante, el componente est� dise�ado para aportar una corriente m�xima al circuito de 1 amperio. Teniendo en cuenta los datos del cuadro \ref{consumo}, solo la BeagleBone Black ya requiere una corriente de 1.2A, y junto al resto de componentes, asciende hasta 1.7A . Un consumo superior al marcado por el fabricante puede producir fallos de funcionamiento en el componente. En este caso particular, podr�a producirse un calentamiento excesivo, que a su vez puede llegar a provocar cortocircuitos internos entre la patilla de entrada y la de salida. Las consecuencias de un fallo as� ser�an nefastas, ya que deteriorar�an los dispositivos conectados a �l.

\subsubsection{Convertidor UBEC}

Un UBEC es un convertidor DC-DC de tipo reductor. El dispositivo se muestra en la figura \ref{ubec}. Los UBEC se utilizan normalmente con bater�as de litio cuando el sistema necesita una tensi�n de alimentaci�n de 5V o 6V. Es por ello que se utilizan mucho en receptores de emisoras de radiocontrol, y gracias a esto existe una amplia variedad de UBECs asequibles en el mercado. La mayor ventaja de estos circuitos convertidores es su eficiencia superior al 90\%. Gracias a esto nuestro sistema tendr� unas perdidas menores, no se calentar� en exceso y gozar� de una autonom�a superior. 

\begin{figure}[h]
\centering
\includegraphics[width=0.5\textwidth]{figuras/ubec}   
\caption{Convertidor DC-DC comercial.}
\label{ubec}
\end{figure}
\FloatBarrier

\subsubsection{Elecci�n final}

Tal y como se comentaba anteriormente, nuestro sistema tiene un consumo estimado de unos 1.7A en la salida de 5V, por lo que un UBEC de 3A ser�a suficiente para proporcionar al circuito la corriente necesaria para su funcionamiento. Adicionalmente se han incluido unos condensadores en la entrada y salida del circuito para evitar ca�das bruscas de tensi�n debidas a requerimientos de intensidad an�malos. Este sistema nos proporcionar� una distribuci�n de alimentaci�n correcta para todos los elementos del robot. %selecci�n de componentes

\chapter{Descripci�n de las herramientas a utilizar}

A continuaci�n se presentan las herramientas b�sicas que ser�n necesarias durante el desarrollo del proyecto.



\section{Herramientas de dise�o y fabricaci�n de piezas}

Dado que la plataforma rob�tica seleccionada requiere modificaciones mec�nicas para permitir el montaje de los componentes necesarios, se requiere construir una serie de piezas que sustituyan a las originales y que aporten al robot algunas caracter�sticas de las que carece. Todas las piezas del robot ser�n dise�adas con OpenSCAD e impresas posteriormente con una impresora 3D open-source.



\subsection{OpenSCAD}

OpenSCAD es un programa destinado a la creaci�n de objetos s�lidos tridimensionales. Se trata de software libre y es compatible con Linux/UNIX, Windows y Mac OS X. A diferencia de otros programas de dise�o 3D, OpenSCAD no se centra en los aspectos art�sticos del dise�o, sino en el aspecto t�cnico. Por ello, es una aplicaci�n muy interesante cuando nuestro objetivo es crear piezas mec�nicas,y en este caso particular, para un robot.

La propiedad mas caracter�stica de OpenSCAD y que le hace diferente de otros programas de dise�o como SolidWorks o FreeCAD, es su interfaz (figura \ref{openscad}). Este programa funciona como un compilador de objetos 3D, leyendo un script qu describe el objeto y renderizando el objeto a partir de ese archivo. Gracias a esto, el usuario tiene total control sobre el proceso de modelado, permitiendo la realizaci�n de modelos variables a partir de par�metros configurables.

OpenCad tambi�n permite el dise�o de modelos planos, siendo compatible con formatos como DXF. Sin embargo, para el caso de este proyecto, los archivos que nos interesa producir son los STL.

\begin{figure}[h]
\centering
\includegraphics[width=1\textwidth]{figuras/openscad}   
\caption{Pantalla del editor de OpenSCAD}
\label{openscad}
\end{figure}



\subsection{Impresi�n 3d}

Para la fabricaci�n de las piezas que integran el robot, se ha utilizado una impresora 3D replicable open-source modelo Prusa i2 Air, de construcci�n casera. La configuraci�n habitual de esta impresora ha sido mejorada con un extrusor Budaschnozzle 1.3 y una electr�nica Sanguinololu 1.3b (figura \ref{impresora}).

Esta impresora, dado su funcionamiento, pertenece a la familia del modelado por deposici�n fundida. La materia prima que utilizan este tipo de impresoras es un rollo de filamento de pl�stico termofusible de entre 1.5 y 3mm de di�metro. En este caso, se utilizar� ABS de 3mm. El filamento de pl�stico es dirigido a un extrusor que empuja el material a trav�s de un conducto caliente conocido como hotend. Al llegar a este punto, el filamento se funde y se hace pasar por un agujero de salida de tama�o muy inferior al de entrada, produciendo hilos de material fundido. Durante la impresi�n, el extrusor deposita pl�stico fundido a lo largo de diferentes trayectorias con el objetivo de formar capas horizontales s�lidas. Mediante la apilaci�n de estas capas se consigue dotar de altura al modelo y crear la pieza requerida.
A trav�s del programa Repetier-Host, que se ocupa de gestionar el funcionamiento de la impresora desde el ordenador, y partiendo de los archivos STL que han sido generados anteriormente desde OpenSCAD, la impresora nos permite desarrollar prototipos y piezas finales para el proyecto.

\begin{figure}[h]
\centering
\includegraphics[width=0.6\textwidth]{figuras/impresora}   
\caption{Impresora 3D Prusa i2 Air}
\label{impresora}
\end{figure}



\section{Herramientas de dise�o de circuitos}

Dado que se va a necesitar expandir las conexiones f�sicas del controlador, se requiere dise�ar una placa de expansi�n que permita realizar un montaje adecuado del sistema.

\subsection{KiCad}

KiCad es una de las herramientas libres m�s avanzadas para el dise�o electr�nico asistido (EDA) que pueden encontrarse a dia de hoy. Cuenta con un grupo de m�s de 150 desarrolladores y una extensa comunidad de usuarios entre quienes se pueden destacar laboratorios como el CERN.

KiCad permite crear dise�os electr�nicos pasando por todas las fases del proceso, desde la idea a los ficheros de fabricaci�n y la simulaci�n 3D de la placa. El entorno (figura \ref{kicad}) cuenta con cuatro aplicaciones independientes:

\begin{itemize}
\item \textbf{Eeschema}. Editor del esquem�tico.
\item \textbf{Pcbnew}. Editor de la placa de circuito impreso.
\item \textbf{Gerbview}. Visor de archivos GERBER
\item \textbf{Cvpcb}. Editor de huellas para componentes.
\end{itemize} 

\begin{figure}[h]
\centering
\includegraphics[width=1\textwidth]{figuras/kicad}   
\caption{KiCad}
\label{kicad}
\end{figure}

KiCad es un programa multiplataforma, escrito con wxWidgets para ser ejecutado en FreeBSD, Linux, Microsoft Windows y Mac OS X. Existe una amplia colecci�n de librer�as de componentes, a las cuales se suma la posibilidad para el usuario de crear componentes personalizados. Adem�s, existen herramientas para importar los componentes de otros EDA, como por ejemplo desde EAGLE. 



\section{Herramientas de programaci�n}

Este proyecto conlleva una programaci�n a diferentes niveles, desde el control de movimientos de los actuadores desde un microcontrolador hasta el dise�o de algortimos de navegaci�n a alto nivel.



\subsection{OpenCV}

OpenCV es una biblioteca libre de visi�n artificial. Fue creada en 1999 por Intel y liberada un a�o mas tarde en la conferencia anual  IEEE Conference on Computer Vision and Pattern Recognition. Existen infinidad de aplicaciones, como la programaci�n de detecci�n de movimiento para c�maras de seguridad o la detecci�n visual de caracter�sticas importantes en diferentes etapas del proceso de fabricaci�n de un producto. La gran popularidad de OpenCV se debe a que su publicaci�n se da bajo licencia BSD, que permite que sea usada libremente para prop�sitos comerciales y de investigaci�n.

OpenCV tambi�n es multiplataforma y existen versiones para GNU/Linux, Windows y Mac OS X. Entre sus funcionalidad, se abarcan campos de la visi�n por computador como el reconocimiento de objetos, calibraci�n de c�maras, visi�n 3D y visi�n rob�tica.
Adem�s, OpenCV est� programado de forma muy optimizada en C y C++, lo que le otorga una alata eficiencia y aptitud para aplicaciones en tiempo real.


\subsection{Qt Creator}

Qt Creator (figura \ref{qt}) es un entorno de desarrollo multiplataforma y open-source desarrollado por la compa��a noruega Trolltech. Soporta C++, JavaScript y QML. El editor incluye resalto de sintaxis y funciones de autocompletado, muy �tiles cuando se trabaja con proyectos de una extensi�n media o alta. Qt Creator utiliza el compilador de C++ de GNU Compiler Collection (o lo que es lo mismo, GCC). Qt Creator intepreta por defecto archivos de CMake.

\begin{figure}[h]
\centering
\includegraphics[width=1\textwidth]{figuras/qt}   
\caption{Qt Creator}
\label{qt}
\end{figure}


\subsection{CMake}

CMake es una herramienta open-source para la administraci�n de la generaci�n de c�digo. El nombre CMake es una abreviatura de cross platform make (make multiplataforma).

CMake es un conjunto de herramientas creado para construir, probar y empaquetar software. Se utiliza para controlar el proceso de compilaci�n del software usando ficheros de configuraci�n sencillos e independientes de la plataforma. CMake genera makefiles nativos y espacios de trabajo que pueden usarse en el entorno de desarrollo deseado. CMake soporta la generaci�n de ficheros para varios sistemas operativos, lo que facilita el mantenimiento y elimina la necesidad de tener varios conjuntos de ficheros para cada plataforma.

El proceso de construcci�n se controla creando uno o m�s ficheros CMakeLists.txt en cada directorio del proyecto.

 
\subsection{CM9 IDE}

CM9 IDE es un entorno de programaci�n basado en Arduino IDE, preparado para programar placas electr�nicas de la serie OpenCM. Soporta programaci�n en C/C++ y es compatible con la mayor�a de las librer�as de Arduino. CM9 (figura \ref{cm9}) constituye un entorno centralizado desde el cual se puede escribir c�digo, compilarlo y cargarlo en la placa OpenCM sin necesidad de ninguna otra aplicaci�n adicional.

\begin{figure}[h]
\centering
\includegraphics[width=1\textwidth]{figuras/cm9}   
\caption{CM9 IDE}
\label{cm9}
\end{figure}

\subsection{Git}

Git es un software de control de versiones dise�ado para controlar el c�digo de un proyecto es su distintas iteraciones. Este proyecto se ha desarrollado en un repositorio online de Github, que ser� liberado cuando TODO (cuando?) 


\paragraph{Palabras clave:} palabraclave1, palabraclave2, palabraclave3.
 %Descripcion de las herramientas a utilizar
  
%chapter introduce un nuevo cap�tulo
\chapter{Desarrollo}

Este proyecto de resume en.................

\paragraph{Palabras clave:} palabraclave1, palabraclave2, palabraclave3.


  
\chapter{Evaluaci�n de resultados}

Se presentn a continuaci�n las conclusiones...

\section{Pruebas de funcionamiento}


\section{Conclusi�n}

Una vez finalizado el proyecto...

\section{Situaci�n y desarrollos futuros}

Un posible desarrollo...

\backmatter   %partes finales del trabajo: bibliografia y anexos



%estilo de bibliograf�a: plana, alfa...
\bibliographystyle{plain}

%genera doble hoja en blanco
\cleardoublepage

%apartado de bibliograf�a
\addcontentsline{toc}{chapter}{Bibliograf�a}

%se incluye la bibliograf�a. Archivo de tipo .bib (bibtex)
\bibliography{bibliografia}

%fin del documento
\end{document}